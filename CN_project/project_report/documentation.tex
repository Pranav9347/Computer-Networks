\documentclass[a4paper,12pt]{article}

\usepackage{graphicx}
\usepackage{hyperref}
\usepackage{amsmath}
\hypersetup{
	colorlinks=true, % Enable colored links
	linkcolor=blue,  % Color for internal links (e.g., table of contents)
	urlcolor=blue,   % Color for external links (URLs)
	citecolor=blue   % Color for citation links
}


\title{Simulation and Analysis of TCP and UDP Performance under Various Network Conditions}
\author{Pranav Vijay Nadgir}
\date{\today}

\begin{document}
	
	\maketitle
	
	\begin{abstract}
		This project focuses on the simulation and analysis of TCP (Transmission Control Protocol) and UDP (User Datagram Protocol) performance under various network conditions. The aim is to evaluate the behavior and performance metrics of each protocol, such as throughput, latency, and packet loss, in a controlled environment. A variety of network scenarios will be created, including different levels of congestion, varying packet sizes, and diverse transmission delays. The tools used for simulation include network simulators like NS3, Wireshark for packet analysis, and scripting languages to automate the testing process. The analysis will provide insights into the suitability of TCP and UDP for specific applications by comparing TCP's reliability and connection-oriented features with UDP's speed and connectionless nature. 
	\end{abstract}
	
	\tableofcontents
	
	\section{Introduction}
	\subsection{Project Objective}
	This project aims to simulate and analyze the performance of TCP and UDP in different network scenarios. The protocols will be compared based on their throughput, latency, and packet loss under various levels of network congestion and delays. The main objective is to understand the trade-offs between reliability and performance for these two widely used transport layer protocols.
	
	\subsection{Motivation}
	Both TCP and UDP are widely used in different applications, but their performance varies under different network conditions. Understanding how each protocol behaves in diverse environments is critical for network engineers and developers to make informed decisions about which protocol to use for specific applications.
	
	\section{Methodology}
	\subsection{Tools and Setup}
	\subsubsection{NS3}
	The network simulator NS3 was used to simulate TCP and UDP traffic under various network conditions. NS3 allows for detailed network modeling, including topology creation, traffic generation, and protocol configuration. The simulation environment was configured to test the protocols in the following conditions:
	\begin{itemize}
		\item Varying levels of network congestion (low, medium, high)
		\item Different packet sizes (small, medium, large)
		\item Various transmission delays (10ms, 100ms, 500ms)
	\end{itemize}
	The following basic NS3 setup was used for the simulation:
	\begin{verbatim}
		Ptr<Socket> tcpSocket = Socket::CreateSocket (nodes.Get (0), TcpSocketFactory::GetTypeId ());
		Ptr<Socket> udpSocket = Socket::CreateSocket (nodes.Get (1), UdpSocketFactory::GetTypeId ());
	\end{verbatim}
	
	\subsubsection{Wireshark}
	Wireshark was used to capture and analyze packet traces generated by the NS3 simulation. The packet loss, throughput, and latency were examined using Wireshark's analysis features. The packet traces were exported in `.pcap` format and analyzed to extract the necessary metrics.
	
	\subsection{Simulation Scenarios}
	The simulation was run under the following conditions:
	\begin{itemize}
		\item Varying levels of congestion (using network traffic generators)
		\item Different packet sizes: small (512 bytes), medium (1500 bytes), large (9000 bytes)
		\item Network delays set to 10ms, 100ms, and 500ms
	\end{itemize}
	The simulations were executed for each combination of these parameters.
	
	\section{Results and Discussion}
	\subsection{Performance Metrics}
	The following performance metrics were collected during the simulation:
	\begin{itemize}
		\item \textbf{Throughput}: The amount of data successfully transferred per unit time.
		\item \textbf{Latency}: The round-trip time for a packet to travel from source to destination and back.
		\item \textbf{Packet Loss}: The percentage of packets that were lost during transmission.
	\end{itemize}
	
	\subsection{Comparison of TCP and UDP}
	\textbf{TCP:} Due to its connection-oriented nature, TCP ensures reliable data transmission with retransmissions and acknowledgment mechanisms. The results indicated that while TCP showed lower packet loss, it suffered from higher latency and reduced throughput under high congestion.
	
	\textbf{UDP:} UDP, being connectionless, provided better throughput and lower latency, especially under light network conditions. However, it exhibited higher packet loss in congested scenarios since it does not have built-in mechanisms for reliability.
	
	The detailed results will be added as the simulations progress.
	
	\section{Conclusion}
	\subsection{Draft Conclusion}
	The comparison between TCP and UDP reveals significant differences in their behavior under various network conditions. TCP’s reliability makes it more suitable for applications that require guaranteed data delivery, such as file transfers or web browsing, while UDP is more appropriate for time-sensitive applications like video streaming and online gaming, where low latency is crucial, and some data loss can be tolerated. Based on the performance metrics collected, recommendations for protocol selection can be made depending on the specific application requirements. Further analysis and testing are required to refine these conclusions and understand the protocols’ behavior in more complex scenarios.
	
	\section{Future Work}
	In the future, we can extend the simulation by:
	\begin{itemize}
		\item Introducing network topology changes (e.g., adding more routers and nodes).
		\item Analyzing additional metrics such as jitter and fairness.
		\item Testing in real-world network environments to validate the simulation results.
	\end{itemize}
	
	\section{References}
	\begin{itemize}
		\item T. S. Eugene and T. E. Johnson, \textit{NS3 Simulation Guide}, Network Simulator 3, 2023.
		\item "Wireshark Official Documentation," Wireshark Foundation, 2024. \url{https://www.wireshark.org/docs/}
		\item Behrouz A Forouzan, \textit{Data Communications and Networking}, 4th Edition.
	\end{itemize}
	
\end{document}
